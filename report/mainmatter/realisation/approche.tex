\section{Réalisation}

Pour implémenter CoCoSpoon nous avons choisi la première technique appelée “Source To Source”\cite{baxter}. Ce choix s'explique de plusieurs manières : il est plus facile de debugger du code source transformé plutôt que le byte code lors du développement. Cependant, les techniques expliquées ci-dessous pourraient être reproduites sur le byte code. \par La différence avec les techniques expliquées en~\ref{sec:principes} est que notre but n'est pas de stocker uniquement de l'information. En effet, afin de pouvoir réaliser le calcul de couverture de code en temps réel, nous devons également interpréter ces informations durant l'exécution du programme. 
