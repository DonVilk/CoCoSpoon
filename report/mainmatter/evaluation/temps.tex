\section{Temps}
\label{temps}

L'évaluation sur le temps d'exécution a été réalisé sur dix exécutions successives du programme. La mesure du temps d'exécution est celle retournée par le framework JUnit lors de l'exécution des tests. La table \ref{evaluation_temps} représente les résultats obtenus. L'overhead s'explique par le fait que dans le pire des cas, le nombre de lignes exécutées est doublé. Le calcul de la vue est également effectué à chaque modification du modèle.

\begin{table}[H]
\centering
\begin{tabular}{|l|c|c|c|}
\hline
         & Commons Lang & Commons Lang' + Vue Text & Commons Lang + JavaFX\\
         \hline
Run 1     & 20.038  & 26.756       & 38.772        \\
Run 2     & 19.605  & 27.020       & 38.450        \\
Run 3     & 19.059  & 26.800       & 38.282        \\
Run 4     & 19.417  & 28.094       & 40.412        \\
Run 5     & 18.810  & 26.016       & 38.834        \\
Run 6     & 20.399  & 27.881       & 39.228        \\
Run 7     & 19.905  & 26.739       & 38.958        \\
Run 8     & 18.737  & 27.293       & 39.102        \\
Run 9     & 18.963  & 26.199       & 39.056        \\
Run 10    & 19.500  & 27.882       & 38.637        \\
\hline
Moyenne  & 19.486  & 27.062       & 38.973        \\
\hline
Overhead (\%) & -  & 38.88       & 100.01   \\
\hline    
\end{tabular}
\caption{Comparatif sur le temps d'exécution en seconde}
\label{evaluation_temps}
\end{table}
